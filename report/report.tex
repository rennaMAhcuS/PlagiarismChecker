\documentclass[11pt]{article}
\usepackage{hyperref}

\usepackage{enumitem}
\setlist[itemize]{left=0pt, itemsep=0pt, topsep=0pt, partopsep=0pt, parsep=0pt}
\setlist[enumerate]{left=0pt, itemsep=0pt, topsep=0pt, partopsep=0pt, parsep=0pt}

\usepackage{xcolor}
\definecolor{RoyalBlue}{RGB}{65, 105, 225}
\renewcommand{\ttdefault}{pcr}
\makeatletter
\let\oldtexttt\texttt
\def\texttt#1{{\color{RoyalBlue}\oldtexttt{#1}}}
\makeatother

\everymath{\color{blue}}

\usepackage{xfrac}
\usepackage{mathpazo}
\usepackage[cal=pxtx]{mathalpha}

% Page dimensions and border setup
\usepackage{geometry}
\geometry{
    a4paper,
    left=8mm,
    right=8mm,
    top=8mm,
    bottom=8mm
}

\usepackage{fontspec}
\setmainfont[
    Path = fonts/,
    UprightFont = *-Regular,
    BoldFont = *-Bold,
    ItalicFont = *-Italic,
    BoldItalicFont = *-BoldItalic
]{Palatino}
\setmonofont[
    Path = fonts/,
    UprightFont = *,
    BoldFont = *,
    ItalicFont = *,
    BoldItalicFont = *,
    Scale=0.9
]{Monaco}

% Title and Authors
\usepackage{eso-pic}
\title{\textbf{Plagiarism Checker}}
\author{
    \textit{Mohana Evuri} \\
    23B1017
    \and
    \textit{Shaik Awez Mehtab} \\
    23B1080
    \and
    \textit{Abhineet Majety} \\
    23B0923
    \\[10pt]
}
\date{\Large \textbf{Fall 2024}}

\pagenumbering{gobble}

\begin{document}

\maketitle

\section{Phase 1}

\subsection{Net Length of Exact Matches Calculation}
\begin{itemize}[noitemsep]
    \item \textbf{Algorithm:} 
    \begin{itemize}
        \item For each matching length ``len'' in the range $\{10, 11, \ldots, 20\}$, the number of matching substrings is counted and added.
        \item The substrings of each length are compared efficiently by \texttt{reinterpret\_cast}, using which a vector of integers is hashed to a sequence of bytes.
    \end{itemize} 
    \item \textbf{Function Signatures:} 
    \begin{itemize}
        \item \texttt{int numExactMatches(const std::vector<int>\&, const std::vector<int>\&)}: Computes the number of exact matches.
    \end{itemize}
\end{itemize}

\subsection{Longest Approximate Match}
\begin{itemize}[noitemsep]
    \item \textbf{Algorithm:} For each pair of start indices of the two vectors, the length of the longest approximate match is found. This is done by simply traversing the vectors. 
    \item \textbf{Function Signatures:}
    \begin{itemize}
        \item \texttt{std::array<int,3> findLongestApproxMatch(std::vector<int>\&, std::vector<int>\&)}: Finds the \\ longest approximate match and the respective start indices.
    \end{itemize}
\end{itemize}

\subsection{Helper Functions}
\begin{itemize}[noitemsep]
    \item \texttt{SubstrMap allSubstrHashes(const std::vector<int>\&, int)}: Hashes all substrings of a given length.
    \item \texttt{std::string getHash(const std::vector<int>\&, int, int)}:
    Hashes a vector of integers.
    \item \texttt{bool is\_already\_checked(const std::vector<bool>\&, int, int)}:
    Checks if a substring was already \\ matched.
    \item \texttt{bool isPlagged(const int, const int, const int, const int)}:
    Given the lengths of submissions, number of matches, and maximum approximate match length, it determines if plagiarism has taken place.
\end{itemize}

\subsection{Overall code flow}
\begin{itemize}[noitemsep]
     
    \item \textbf{Data Structures:} We have used \textbf{vector} and \textbf{unordered multimap} in our implementation.
    \item \textbf{Flow:} \begin{itemize}
        \item When \texttt{match\_submissions} is called, it calls \texttt{numExactMatches}, \texttt{longestApproxMatch} and \texttt{isPlagged} in that order
        \item \texttt{numExactMatches} uses \texttt{isAlreadyChecked} and \texttt{allSubstrHashes}, which uses \texttt{getHash}.
    \end{itemize}
\end{itemize}

\subsection{Complexity Analysis}
Let $n$ and $m$ be the submission sizes.
\begin{itemize}[noitemsep]
    \item \textbf{Time Complexity:} $O(nm\cdot\min(n, m))$: Calculating the number of exact matches takes $O(nm)$ time. Finding the longest approximate matching substring takes $O(nm\cdot\min(n, m))$ time.
    \item \textbf{Space Complexity:} $O(n + m)$: The calculation of the number of exact matches takes $O(n + m)$ space. Finding the longest approximate matching substring takes $O(1)$ space.
\end{itemize}

\section{Phase 2}
\vspace{-5pt}
\subsection{Plagiarism Detection}
\begin{itemize}[noitemsep]
    \item \textbf{Algorithm:} Dynamic programming is being used. The length of the matching substring that ends in the indices is recorded for each pair of indices $i, j$ of the vectors.
    \begin{itemize}
        \item \textbf{Short Pattern Match Detection}: Whenever the length of the matching substring equals 15, the number of matches is incremented.
        \item \textbf{Long Pattern Match Detection}: For each pair of indices, the length of the longest match is updated if needed.
        \item \textbf{Patchwork Plagiarism Detection}: The number of matches with submissions made in the last second is recorded for each submission. Flagging is done based on this measure.
    \end{itemize}
    \item \textbf{Function Signatures:}
    \begin{itemize}
        \item \texttt{void check\_two\_submissions(std::pair<int, std::shared\_ptr<submission\_t>>, \\ std::pair<int, std::shared\_ptr<submission\_t>>, int, int, int\&)}:
        Checks if two submissions are similar and flags them if necessary.
        \item \texttt{void plagiarism\_checker\_t::process\_submissions(int)}:
        Compares the current submission with all previous submissions and flags in case of patchwork plagiarism.
    \end{itemize}
\end{itemize}
\vspace{-5pt}
\subsection{Complexity Analysis}
Let the number of files be $m$ and the average number of tokens per file be $n$.
\begin{itemize}[noitemsep]
    \item \textbf{Time Complexity:} $O(mn^2)$: Comparing two submissions takes $O(n^2)$ time, and there are $O(m)$ files to compare for each submission.
    \item \textbf{Space Complexity:} $O(mn)$: A cache of tokens of all previous submissions is maintained, which takes $O(mn)$ space. Vectors and queues take $O(m)$ space.
\end{itemize}
\vspace{-5pt}
\subsection{Concurrency Features}
\begin{itemize}[noitemsep]
    \item \textbf{Threading:} Threading is used to make \texttt{add\_submission} non-blocking and process submissions parallelly:
    \begin{itemize}
        \item The function \texttt{add\_submission} pushes the submission to a queue. Submissions are processed in \\ The \texttt{processor} thread. Submission is made non-blocking in this way.
        \item The \texttt{processor} thread maintains the metadata in \texttt{to\_check} and initiates tasks in \texttt{pipe} queue. The thread \texttt{processor} performs the actual processing of comparing and flagging submissions.
    \end{itemize}
    \item \textbf{Concurrency:} 
    \begin{itemize}
        \item Concurrency is maintained using \texttt{mutex} and locks to avoid data races when shared data is used.
        \item Conditional variable \texttt{cv} sends signals so that thread waits passively.
        \item  The atomic variable \texttt{stop} is used to terminate the code properly in the presence of threading.
    \end{itemize} 
\end{itemize}
\vspace{-5pt}
\subsection{File Identification}
When a submission is made, it is checked against each file submitted before it via the vector \texttt{to\_check}. If a submission was made within one second before the submission, even that is flagged if it was not already flagged. The vector maintains every submission's pointer, timestamp, and status(flagged/not flagged).
\vspace{-5pt}
\subsection{Helpers and Data Structures}
\begin{itemize}[noitemsep]
    \item \textbf{Helpers:}
    \item \begin{itemize}
        \item \texttt{void plagiarism\_checker\_t::process\_submissions()}:
        Updates metadata of each new submission.
        \item \texttt{int plagiarism\_checker\_t::curr\_time\_millis()}:
        Gives the current time according to the \\ \texttt{system\_clock}.
    \end{itemize}
    \item \textbf{Data Structures:} The data structures \textbf{queue} and \textbf{vector} are used.
    \item \textbf{Flow:} 
    \begin{itemize}
        \item In the constructor, threads are started, and the vector is initialized with initial submissions.
        \item When \texttt{add\_submission} is called, the submission is pushed to a queue.
        \item Processing is done by the other thread, leaving the main thread free to add other submissions.
        \item When the destructor is called, threads are notified that there will be no more submissions and joined to the main thread.
    \end{itemize}
\end{itemize}

\end{document}
